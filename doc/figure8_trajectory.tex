\documentclass{article}
\usepackage{amsmath, amssymb, amsthm, graphicx}
\usepackage{geometry}
\geometry{margin=1in}

\title{Figure-8 Trajectory Generation in Robotics}
\author{Spencer Maughan}
\date{March 2025}

\begin{document}

	\maketitle

	\section{Introduction}
	A smooth and well-defined trajectory is crucial for testing and evaluating state estimation models, measurement systems, and prediction algorithms in robotics. This document outlines the design and implementation of a \textbf{Figure-8 Trajectory Generator}, which provides a robust motion profile for dynamic testing scenarios.

	\section{Trajectory Configuration}
	The Figure-8 trajectory is configured using a set of well-defined parameters:

	\begin{itemize}
		\item \textbf{Maximum Velocity} ($v_{\text{max}}$): Controls the peak velocity along the trajectory.
		\item \textbf{Length} ($a$): Specifies the amplitude of the X-axis motion.
		\item \textbf{Width} ($b$): Specifies the amplitude of the Y-axis motion.
		\item \textbf{Width Slope} ($c$): Determines the inclination in the Z-direction.
		\item \textbf{Angular Scale} ($\alpha$): Scales the angular motion for orientation dynamics.
	\end{itemize}

	\section{Mathematical Formulation}
	The position in the Figure-8 trajectory is parameterized by time as follows:

	\[
	\mathbf{p}(t) =
	\begin{bmatrix}
		a \cos(w_2 t) \\
		b \sin(w_1 t) \\
		c \sin(w_1 t)
	\end{bmatrix}
	\]

	Where:
	\begin{itemize}
		\item $a = \frac{\text{length}}{2}$ (X amplitude)
		\item $b = \frac{\text{width}}{2}$ (Y amplitude)
		\item $c = \text{width} \times \tan(\text{width slope})$ (Z amplitude)
		\item $w_1 = \frac{2\pi}{T}$ (frequency for Y and Z motion)
		\item $w_2 = w_1 \times 0.5$ (frequency for X motion)
		\item $T = \frac{\pi \sqrt{a^2 + 4(b^2 + c^2)}}{v_{\text{max}}}$ (Trajectory period)
	\end{itemize}

	\section{Orientation and Angular Dynamics}

	To ensure the body frame’s X-axis aligns with the velocity direction, the orientation is defined using a quaternion. The axis of rotation is computed as the cross product of the inertial velocity direction and the X-axis in the world frame:

	\[
	\mathbf{r} = \hat{x} \times \frac{\mathbf{v}}{||\mathbf{v}||}
	\]

	The angle of rotation is the angle between these two vectors:

	\[
	\theta = \cos^{-1}\left( \hat{x} \cdot \frac{\mathbf{v}}{||\mathbf{v}||} \right)
	\]

	The corresponding quaternion representation is constructed as:

	\[
	\mathbf{q} = \left[\cos\left(\frac{\theta}{2}\right),\, \mathbf{r} \sin\left(\frac{\theta}{2}\right)\right]
	\]

	This quaternion is normalized to ensure numerical stability:

	\[
	\mathbf{q} = \frac{\mathbf{q}}{||\mathbf{q}||}
	\]

	The velocity in the body frame is calculated by transforming the inertial velocity vector:

	\[
	\mathbf{v}_{\text{body}} = \mathbf{q}^{-1} \mathbf{v}_{\text{inertial}}
	\]

	The angular velocity in the world frame is calculated as:

	\[
	\boldsymbol{\omega}_{\text{world}} = \frac{\mathbf{v} \times \mathbf{a}}{||\mathbf{v}||^2}
	\]

	The angular velocity is then converted to the body frame:

	\[
	\boldsymbol{\omega}_{\text{body}} = \mathbf{q}^{-1} \boldsymbol{\omega}_{\text{world}}
	\]

	The instantaneous roll rate is the component of angular velocity about the body-frame X-axis:

	\[
	\dot{\phi} = \hat{x}_{\text{body}} \cdot \boldsymbol{\omega}_{\text{body}}
	\]

	The body-frame acceleration is adjusted to incorporate full Coriolis compensation:

	\[
	\mathbf{a}_{\text{body}} = \mathbf{q}^{-1} \mathbf{a} - 2 \boldsymbol{\omega}_{\text{body}} \times \mathbf{v}_{\text{body}}
	\]

	The angular acceleration is derived as:

	\[
	\boldsymbol{\alpha}_{\text{world}} = \frac{\mathbf{v} \times \mathbf{j} \cdot ||\mathbf{v}||^2 - 2\mathbf{v} \times \mathbf{a} \cdot \mathbf{v} \cdot \mathbf{a}}{||\mathbf{v}||^4}
	\]

	Finally, the angular acceleration is transformed to the body frame:

	\[
	\boldsymbol{\alpha}_{\text{body}} = \mathbf{q}^{-1} \boldsymbol{\alpha}_{\text{world}}
	\]

	The roll acceleration is similarly defined as the rate of change of angular velocity about the body-frame X-axis:

	\[
	\ddot{\phi} = \hat{x}_{\text{body}} \cdot \boldsymbol{\alpha}_{\text{body}}
	\]

	\section{Conclusion}
	This Figure-8 trajectory generator offers a robust and versatile tool for testing state estimation, measurement models, and prediction systems. By combining smooth position dynamics with controlled angular motion, it effectively replicates challenging real-world motion patterns for robotics testing environments.

\end{document}
