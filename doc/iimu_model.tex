\documentclass{article}
\usepackage{amsmath, amssymb, xcolor, listings, booktabs, geometry}
\definecolor{codeblue}{RGB}{25,25,112}
\geometry{a4paper, margin=1in}

\begin{document}

	\section*{IMU Measurement Model with Sensor Offset and Misalignment}

	We consider an IMU mounted at a known offset $\mathbf{r}$ from the body reference frame origin, with a known fixed rotation $\mathbf{R}_{BS}$ from body frame ($B$) to sensor frame ($S$). The IMU provides measurements of linear acceleration and angular velocity. The 19-dimensional state vector is defined as:
	\[
	\mathbf{x} = \begin{bmatrix}
		\mathbf{p} \\[4pt]
		\mathbf{q} \\[4pt]
		\mathbf{v} \\[4pt]
		\boldsymbol{\omega} \\[4pt]
		\mathbf{a} \\[4pt]
		\boldsymbol{\alpha}
	\end{bmatrix}
	=
	\begin{bmatrix}
		p_x, p_y, p_z \\[4pt]
		q_0, q_1, q_2, q_3 \\[4pt]
		v_x, v_y, v_z \\[4pt]
		\omega_x, \omega_y, \omega_z \\[4pt]
		a_x, a_y, a_z \\[4pt]
		\alpha_x, \alpha_y, \alpha_z
	\end{bmatrix},
	\]
	where $\mathbf{q}$ is the quaternion representing orientation from inertial to body frame.

	The IMU provides two measurements:

	\begin{enumerate}
		\item Accelerometer measurement ($3\times1$):
		$$
		\mathbf{a}_{IMU} = \mathbf{R}_{BS}\left[\mathbf{a} + \mathbf{R(q)}^T \mathbf{g} + \boldsymbol{\alpha}\times \mathbf{r} + \boldsymbol{\omega}\times(\boldsymbol{\omega}\times \mathbf{r})\right]
		$$

		\item Gyroscope measurement ($3\times1$):
		$$
		\boldsymbol{\omega}_{IMU} = \mathbf{R}_{BS}\,\boldsymbol{\omega}
		$$
	\end{enumerate}

	Here:
	- $\mathbf{R(q)}^T$ rotates vectors from inertial to body frame.
	- $\mathbf{R}_{BS}$ is a constant rotation matrix from body to sensor frame (misalignment).
	- $\mathbf{g}$ is gravitational acceleration in inertial coordinates.
	- $\mathbf{r}$ is lever arm vector (body coordinates).

	\vspace*{10pt}

	\section*{Jacobian Derivation}

	The Jacobian matrix $\mathbf{H}$ is defined as:
	$$
	\mathbf{H} =
	\frac{\partial}{\partial \mathbf{x}}
	\begin{bmatrix}
		\mathbf{a}_{IMU}\\[6pt]
		\boldsymbol{\omega}_{IMU}
	\end{bmatrix}_{6\times1},
	$$
	thus it has dimensions $6\times19$.

	We compute each block separately:

	\vspace*{10pt}

	\subsection*{Accelerometer Measurement Jacobians ($3\times19$)}

	- With respect to linear position ($3\times3$):
	$$
	\frac{\partial \mathbf{a}_{IMU}}{\partial \mathbf{p}} =
	\mathbf{0}_{3\times3}.
	$$

	- With respect to quaternion orientation ($3\times4$), gravity term only:
	$$
	J_q = \frac{\partial (\mathbf{R}_{BS}\,\mathbf R(q)^T\, g)}{\partial q} =
	\mathbf{R}_{BS}\frac{\partial (\mathbf R(q)^T g)}{\partial q},
	$$
	where:
	$$
	\frac{\partial (\mathbf R(q)^T g)}{\partial q}=2\,[
	(q_0 g + \tilde q\times g),~~(\tilde q\,g^T+(q_0 I+[\tilde q]_\times)[g]_\times))]
	_{3\times4},
	$$
	with quaternion $q=[q_0,\tilde q^T]^T$, and $[\tilde q]_\times$ skew-symmetric operator.

	- With respect to linear velocity ($3\times3$):
	$$
	\frac{\partial \mathbf a_{IMU}}{\partial \mathbf v}=0_{3\times 3}.
	$$

	- With respect to angular velocity ($3\times3$):
	Define: $h(\boldsymbol{\omega})=\boldsymbol{\omega}\times(\boldsymbol{\omega}\times r)$,
	then:
	$$
	\frac{\partial h}{\partial \boldsymbol{\omega}}=[(\boldsymbol{\omega}\times r)]_\times-[r]_\times[\boldsymbol{\omega}]_\times.
	$$

	Thus:
	$$
	\frac{\partial \mathbf a_{IMU}}{\partial {\boldsymbol {\omega}} }=\mathbf R_{BS}\left([(\boldsymbol{\omega}\times r)]_\times-[r]_\times[\boldsymbol{\omega}]_\times\right).
	$$

	- With respect to linear acceleration ($3\times3$):
	Direct identity mapping:
	$$
	\frac{\partial {\bf a}_{IMU}}{\partial {\bf a}}=\bf R_{BS}.
	$$

	- With respect to angular acceleration ($3\times3$):
	Tangential acceleration depends linearly:
	$$
	\frac{\partial ({\boldsymbol {\alpha}}\times r)} {\partial {\boldsymbol {\alpha}} }=\mathbf R_{BS}[r]_{\times}.
	$$

	Thus accelerometer Jacobian block row is:
	$$
	H_{accel}=
	[\underbrace{{0}_{(3\times 3)}}_{\textbf p},~
	~J_q^{grav}(q)_{(3\times4)},~
	~0_{(3\times 3)},~
	~\mathbf R_{BS}\left([(\boldsymbol{\omega}\times r)]_\times-[r]_\times[\boldsymbol{\omega}]_\times\right),~
	~R_{BS},~~
	~R_{BS}[r]_{\times}]
	$$

	\vspace*{10pt}

	\subsection*{Gyroscope Measurement Jacobians ($3\times19$)}

	Only angular velocity states affect gyroscope directly:
	$$
	H_{gyro}=[\,0_{(3\times 3)},~~0_{(3\times 4)},~~0_{(3\times 3)},~~R_{BS},~~0_{(3\times 3)},~~0_{(3\times 3)}\,].
	$$

	\vspace*{10pt}

	Thus the full Jacobian matrix (6$\times$19) is clearly given by stacking both blocks:

	$$
	H=
	\begin {bmatrix}
	0 & J_q^{grav}& 0 & R_{BS}\left([(\boldsymbol{\omega}\times r)]_\times-[r]_\times[\boldsymbol{\omega}]_\times)\right)& R_{BS}& R_{BS}[r]_{\times}\\[8pt]
	0 & 0 & 0 & R_{BS}& 0 & 0
	\end {bmatrix}.
	$$

	This final form correctly incorporates both sensor offset and misalignment rotation.

\section{IMU Quaternion Jacobian Derivation \& Validation}
\label{sec:theory}

\subsection{Problem Definition}
For an IMU with orientation represented by unit quaternion $\mathbf{q} = [q_w, q_x, q_y, q_z]^T$, we derive the Jacobian for the gravity compensation term:
\begin{equation}
	\mathbf{f(q)} = \mathbf{R_{BS}R(q)^T g_W}
\end{equation}
where $\mathbf{g_W} = [0, 0, g]^T$ is gravity in world coordinates, and $\mathbf{R_{BS}}$ is the fixed body-to-sensor rotation.

\subsection{Analytical Derivation}
\label{subsec:analytical}

The rotation matrix $\mathbf{R(q)^T}$ for a unit quaternion is:
\begin{equation}
	\mathbf{R(q)^T} =
	\begin{bmatrix}
		1-2(q_y^2+q_z^2) & 2(q_xq_y+q_wq_z) & 2(q_xq_z-q_wq_y) \\
		2(q_xq_y-q_wq_z) & 1-2(q_x^2+q_z^2) & 2(q_yq_z+q_wq_x) \\
		2(q_xq_z+q_wq_y) & 2(q_yq_z-q_wq_x) & 1-2(q_x^2+q_y^2)
	\end{bmatrix}
\end{equation}

The body-frame gravity vector and its Jacobian:
\begin{equation}
	\mathbf{R(q)^T g_W} = g\begin{bmatrix}
		2(q_xq_z - q_wq_y) \\
		2(q_yq_z + q_wq_x) \\
		1 - 2(q_x^2 + q_y^2)
	\end{bmatrix}, \quad
	J_q^{\text{grav}} = 2g\begin{bmatrix}
		-q_y & q_z & -q_w & q_x \\
		q_x & q_w & q_z & q_y \\
		0 & -2q_x & -2q_y & 0
	\end{bmatrix}
\end{equation}

\section{Numerical Validation Methodology}
\label{sec:validation}

\subsection{Core Principles}
\begin{itemize}
	\item Maintain unit quaternion constraint $\|\mathbf{q}\|=1$
	\item Use consistent frame conventions (ROS REP 103 recommended)
	\item Validate identity case before complex orientations
\end{itemize}

\subsection{Perturbation Technique}
\label{subsec:perturb}

\subsubsection{Exponential Map Formulation}
For valid quaternion perturbations:
\begin{equation}
	\delta\mathbf{q} = \exp\left(\frac{1}{2}\delta\boldsymbol{\theta}\right) \approx \begin{bmatrix}
		1 \\
		\delta\theta_x/2 \\
		\delta\theta_y/2 \\
		\delta\theta_z/2
	\end{bmatrix}, \quad \|\delta\boldsymbol{\theta}\| \ll 1
\end{equation}

Implementation code:
\begin{lstlisting}[language=C++,basicstyle=\color{codeblue}\ttfamily]
	const double eps = 1e-6; // Optimal for double precision
	Eigen::Vector3d delta_theta(eps, 0, 0); // X-axis perturbation

	Eigen::Quaterniond dq(1, delta_theta.x()/2,
	delta_theta.y()/2,
	delta_theta.z()/2);
	dq.normalize();
	Eigen::Quaterniond q_perturbed = q * dq;
\end{lstlisting}

\subsection{Validation Protocol}
\label{subsec:protocol}

\begin{enumerate}
	\item \textbf{Baseline Check}: Identity quaternion $\mathbf{q}=[1,0,0,0]^T$
	\item \textbf{Canonical Rotations}: 90° about each principal axis
	\item \textbf{Arbitrary Orientation}: Random unit quaternion
	\item \textbf{Error Analysis}:
	\begin{equation}
		\varepsilon_{\text{rel}} = \frac{\|J_{\text{ana}} - J_{\text{num}}\|}{\max(\|J_{\text{ana}}\|, \epsilon)}
	\end{equation}
	\begin{itemize}
		\item Accept: $\varepsilon_{\text{rel}} < 1\%$
		\item Investigate: $1\% \leq \varepsilon_{\text{rel}} \leq 5\%$
		\item Reject: $\varepsilon_{\text{rel}} > 5\%$
	\end{itemize}
\end{enumerate}

\section{Case Study: 90° Y-Rotation}
\label{sec:case_study}

\subsection{Test Configuration}
\begin{itemize}
	\item Quaternion: $[0.7071, 0, 0.7071, 0]^T$ (90° Y-rotation)
	\item Gravity: $g = 9.80665\,\text{m/s}^2$
	\item Perturbation: $\delta\theta = 1\mu\text{rad}$
\end{itemize}

\subsection{Results \& Analysis}
\begin{tabular}{llll}
	Term & Analytical & Numerical & Error \\
	\midrule
	$\partial g_x/\partial q_w$ & -13.87 & -13.87 & 0.00\% \\
	$\partial g_z/\partial q_y$ & -27.74 & -27.74 & 0.00\% \\
	$\partial g_y/\partial q_z$ & 13.87 & 13.87 & 0.00\% \\
\end{tabular}

\section{Best Practices \& Troubleshooting}
\label{sec:best_practices}

\subsection{Implementation Checklist}
\begin{itemize}
	\item [$\square$] Use exponential map perturbations
	\item [$\square$] Normalize after quaternion operations
	\item [$\square$] Verify $\mathbf{R_{BS}}$ separately
	\item [$\square$] Test positive/negative perturbations
	\item [$\square$] Check magnitude and sign
\end{itemize}

\subsection{Common Issues \& Solutions}
\begin{tabular}{p{4.5cm}p{8cm}}
	\textbf{Symptom} & \textbf{Resolution} \\
	\midrule
	Sign mismatches & Verify quaternion multiplication order (Hamilton vs JPL) \\
	Null Z-derivatives & Confirm gravity vector alignment in world frame \\
	2x error scaling & Check 1/2 factor in exponential map \\
	Discontinuities at identity & Use $\mathbf{q}=[0.999,0,0,0.001]$ near identity \\
	Frame inconsistencies & Validate $\mathbf{R_{BS}}$ matrix independently \\
	Sensitivity to perturbation size & Test $\epsilon\in[10^{-7},10^{-4}]$ \\
\end{tabular}

\appendix
\section{Reference Implementation}
\label{app:code}

Complete Jacobian computation:
\begin{lstlisting}[language=C++,basicstyle=\color{codeblue}\ttfamily]
	Eigen::Matrix<double, 3, 4> ComputeQuatJacobian(
	const Eigen::Quaterniond& q,
	double g,
	const Eigen::Matrix3d& R_BS)
	{
		Eigen::Matrix<double, 3, 4> J;
		const double two_g = 2 * g;

		J << two_g*q.y(),  two_g*q.z(),  two_g*q.w(),  two_g*q.x(),
		-two_g*q.x(), -two_g*q.w(),  two_g*q.z(),  two_g*q.y(),
		0.0, -4*g*q.x(),   -4*g*q.y(),            0.0;

		return R_BS * J;
	}
\end{lstlisting}


	\end {document}
