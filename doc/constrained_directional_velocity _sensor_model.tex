\documentclass{article}
\usepackage{amsmath,amssymb,braket}

\begin{document}

	\title{Directional Velocity Constraint in Sensor-Model Kalman Filter}
	\maketitle

	\section{Problem Definition}
	Estimate a 19D state vector containing position, orientation (quaternion), velocities, and accelerations using:
	\begin{itemize}
		\item Asynchronous position measurements $\mathbf{z}_p \in \mathbb{R}^3$
		\item Scalar velocity magnitude $z_v \in \mathbb{R}$ aligned with heading
	\end{itemize}

	\section{State Representation}
	Define the state vector:
	\[
	\mathbf{x} = \begin{bmatrix}
		\mathbf{p} \\ \mathbf{q} \\ \mathbf{v} \\ \boldsymbol{\omega} \\ \mathbf{a} \\ \boldsymbol{\alpha}
	\end{bmatrix} = \begin{bmatrix}
		x & y & z & q_w & q_x & q_y & q_z & v_x & v_y & v_z & \cdots\\
	\end{bmatrix}^T
	\]
	where $\mathbf{q} = (q_w, q_x, q_y, q_z)$ is a unit quaternion.

	\section{Measurement Model}
	\subsection{Position Measurement}
	\[
	\mathbf{z}_p = \mathbf{p} + \boldsymbol{\nu}_p,\quad \boldsymbol{\nu}_p \sim \mathcal{N}(0,\mathbf{R}_p)
	\]
	Jacobian: $\mathbf{H}_p = \begin{bmatrix}\mathbf{I}_3 & \mathbf{0}_{3\times16}\end{bmatrix}$

	\subsection{Velocity-Direction Constraint}
	Define heading vector from quaternion:
	\[
	\mathbf{h}(\mathbf{q}) = \mathbf{R}(\mathbf{q})\begin{bmatrix}1\\0\\0\end{bmatrix} = \begin{bmatrix}
		1-2(q_y^2 + q_z^2) \\
		2(q_xq_y + q_wq_z) \\
		2(q_xq_z - q_wq_y)
	\end{bmatrix}
	\]
	Velocity measurement model:
	\[
	z_v = \mathbf{v}^T\mathbf{h}(\mathbf{q}) + \nu_v,\quad \nu_v \sim \mathcal{N}(0,R_v)
	\]

	\section{Constraint Jacobian}
	Compute $\mathbf{H}_v = \frac{\partial}{\partial\mathbf{x}}(\mathbf{v}^T\mathbf{h}(\mathbf{q}))$:

	\subsection{Velocity Components}
	\[
	\frac{\partial z_v}{\partial\mathbf{v}} = \mathbf{h}(\mathbf{q})^T = \begin{bmatrix}
		1-2(q_y^2+q_z^2) & 2(q_xq_y+q_wq_z) & 2(q_xq_z-q_wq_y)
	\end{bmatrix}
	\]

	\subsection{Quaternion Components}
	\[
	\frac{\partial z_v}{\partial\mathbf{q}} = \mathbf{v}^T\frac{\partial\mathbf{h}}{\partial\mathbf{q}} = \begin{bmatrix}
		2(v_yq_z - v_zq_y) \\
		2(v_yq_y + v_zq_z) \\
		-4v_xq_y + 2v_yq_x - 2v_zq_w \\
		-4v_xq_z + 2v_yq_w + 2v_zq_x
	\end{bmatrix}^T
	\]

	Full Jacobian structure:
	\[
	\mathbf{H}_v = \begin{bmatrix}
		\mathbf{0}_{1\times3} & \frac{\partial z_v}{\partial\mathbf{q}} & \frac{\partial z_v}{\partial\mathbf{v}} & \mathbf{0}_{1\times9}
	\end{bmatrix}
	\]

	\section{Filter Implementation}
	\subsection{Prediction Step}
	Propagate state using high-order dynamics:
	\[
	\dot{\mathbf{x}} = f(\mathbf{x},\mathbf{u}) \quad \text{(Physics model)}
	\]
	with quaternion propagation:
	\[
	\dot{\mathbf{q}} = \frac{1}{2}\mathbf{q} \otimes \begin{bmatrix}0 \\ \boldsymbol{\omega}\end{bmatrix}
	\]

	\subsection{Update Steps}
	Handle measurements asynchronously:
	\begin{align*}
		\text{Position Update:} & \quad \text{Apply } \mathbf{H}_p \text{ with Kalman gain} \\
		\text{Velocity Update:} & \quad \text{Apply } \mathbf{H}_v \text{ using current state estimate}
	\end{align*}

	\section{Implementation Notes}
	\begin{itemize}
		\item \textbf{Quaternion Normalization:} Renormalize $\mathbf{q}$ after each update
		\[
		\mathbf{q} \leftarrow \mathbf{q}/\|\mathbf{q}\|
		\]
		\item \textbf{Adaptive Noise:} Scale $R_v$ inversely with $\|\mathbf{v}\|$
		\[
		R_v \leftarrow R_0(1 + \|\mathbf{v}\|^{-1})
		\]
	\end{itemize}

	\section{Conclusion}
	This sensor-model constraint approach:
	\begin{itemize}
		\item Maintains high-order state dynamics integrity
		\item Enforces velocity-heading alignment through measurement residuals
		\item Handles asynchronous sensors naturally
		\item Requires only 19D matrix operations (sparse Jacobians)
	\end{itemize}

\end{document}
